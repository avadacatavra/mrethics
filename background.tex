Tech companies have embraced the mantras of 'ask for forgiveness not permission' and 'move fast and break things' to the detriment of individuals' privacy, security, and safety. Instead of thoughtfully approaching difficult problems and considering how we can prevent abuse, we sell applications that actively aid abusers \cite{dell}. Instead of designing systems to empower and protect users, we create environments that foster harassment without clear or sufficient accountability mechanisms \cite{dreyfuss}. Instead of debiasing algorithms and training data, we build self-driving cars that are more likely to hit dark skinned people\cite{wilson2019predictive}.

By building technology that violates users' privacy and denies them agency, we're creating a dystopian future. Mixed reality, with its ability to combine virtual and physical elements, is a powerful mechanism for distorting our perspectives.  In 1987, Simitis argued that large scale data collection is "the ideal means to adapt an individual to a predetermined, standardized behavior that aims at the highest possible degree of compliance with the model patient, consumer, taxpayer, employee, or citizen." \cite{simitis} If we extend this reasoning to the scale of data collection today and then consider incorporating MR-derived data, it's clear that we need to embrace ethical principles before it's too late.

Generally, the ethics of emerging technologies are focused on ethical assessments of research and innovation. Mixed reality (MR) ethics occupies a space that encompasses emerging tech ethics, healthcare ethics and product ethics. The technology is beyond emerging, but not quite entrenched. We're still in a position to intervene in the development process, instead of attempting to retrofit ethical decisions into an established design. 

This paper focuses on issues that face building a platform that encourages ethical development and usage. While we have a wider range of data to analyze than most emerging technologies, we're still in a much more speculative state than entrenched technologies. This space is a challenge and an opportunity.

\subsection{Web ethics}

For our purposes, there's also an overlap with web ethics, because we're building a platform for the immersive web. Having an open and accessible web means that we can invite more diverse viewpoints. Today, if you want to be an iOS or Android app author, you must be in an approved nation; otherwise, you and your country can't participate in the ecosystem.

The web is shaped by standards bodies like the World Wide Web Consortium (W3C). Standards are crafted by consensus, so the intentions of a single bad actor are minimized.

When one browser has a monopoly, developers aren't incentivized to make their sites work on multiple platforms. This decreases both competition and the efficacy of web standards---if there's only one major browser, then their implementation becomes synonymous with the standard. We need diverse viewpoints and interests to shape the web, otherwise it will only serve the interests of a few.

The immersive web has a number of advantages over a solely app-based ecosystem. Unlike apps, there are no inherent restrictions on who can develop or access web resources. It's also intended to be cross-platform, allowing users with a \$300 MR device to have a similar experience to those with a \$3000 device. Perhaps most importantly, it allows web browsers to act as a trusted intermediary for device resource requests. Instead of a native app running in the background with access to information like orientation data, the webpage needs to request this through the browser, which could reject inappropriate requests.


\subsection{Mixed reality ethics}

Mixed reality technology has the potential to transform the way we interact with each other and the world around us. The best way to level out asymmetries of knowledge and power is to not allow them to form in the first place. This paper is an initial exploration into the challenges we face while we try to define what an ethical immersive future looks like. I propose the following principles of building ethical software in mixed reality:

\begin{itemize}
	\item Ask permission, not forgiveness
	\item Minimize tracking and fingerprinting via biometrics
	\item Empower individuals to define how they're perceived virtually
	\item Prioritize mechanisms for reporting harassment and blocking perpetrators
	\item Identify ways to incentivize the principle of least privilege
	\item Consider privacy a first-class requirement
	\item Be both transparent and accountable
\end{itemize}


Section \ref{sec:laws} briefly discusses relevant laws, and section \ref{sec:steps} proposes concrete steps we can take to embrace that these principles.
