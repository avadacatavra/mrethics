With the advent of affordable devices like Oculus Go and augmented reality-enabled smartphones, mixed reality devices are hitting the mainstream market. Market growth projections over the next five years range from 40 to 80\%.\footnote{\url{https://medium.com/vr-first/a-summary-of-augmented-reality-and-virtual-reality-market-size-predictions-4b51ea5e2509}} As we have experienced with the ubiquity of artificial intelligence (AI) and machine learning (ML), the development of ethical frameworks and guidelines tends to lag behind the technology itself. Once we begin to fully recognize the impact of new technologies, we are left to retrofit regulations and ethical decision making into technology that has already had billions of dollars invested in it.
% https://medium.com/vr-first/a-summary-of-augmented-reality-and-virtual-reality-market-size-predictions-4b51ea5e2509

Mixed reality (MR) devices blend digital elements and the physical world, covering a wide spectrum that includes both virtual reality (VR) and augmented reality (AR). In VR, a device occludes the user's vision (and often other senses) to present a fully digital experience, while AR experiences overlay digital elements on users' perceptions of the physical world.\footnote{Two other terms that often appear when discussing MR are \emph{spatial computing} and \emph{immersive technologies}.} 

In this work, MR ethics overlaps with web ethics (i.e., the process of making ethical decisions for the foundations of the web), as the goal is to build a platform for the immersive web. Having an open and accessible web means that we can invite more diverse viewpoints. Today, if you want to be an contribute to the web environment, for example, as an iOS or Android app author, you must be in an approved nation; otherwise, you and your country can not participate in the ecosystem.

% https://docs.microsoft.com/en-us/windows/mixed-reality/mixed-reality
The contributions of this paper are to outline the unique ethical and privacy risks posed by MR devices, present plausible scenarios that may result from insufficient protections, and propose steps that will improve protections from legal, regulatory, societal, and engineering standpoints. This paper is intended to provide a starting point for making ethical decisions when creating and engineering MR technologies with a particular focus on the importance and practice of privacy. It will distill the discussions that are occurring in the larger privacy and ethics community aimed at informing those creating the next generation of MR technology.

% TODO restructure this after the rewrite
Section \ref{sec:background} provides an overview of why ethics matter and introduces the types of data that MR experiences process and collect. In \autoref{sec:studies}, we present discussion and analysis of MR application areas, framed around ethical scenarios and case studies, which motivates seven principles presented in \autoref{sec:principle}. Finally, \autoref{sec:discussion} overviews the current legal landscape for MR technology and provides concrete steps technologists may take to create a more ethical immersive future.

This paper focuses on issues that face building a platform that encourages ethical development and usage. The ideas in this paper can be used by app and platform developers to consider the ethical implications of individual apps, and create platforms that encourage ethical choices when developing these apps. While we have a wider range of data to analyze than most emerging technologies, we are still in a much more speculative state than entrenched technologies. This is both a challenge and an opportunity.


% TODO work in immersive web