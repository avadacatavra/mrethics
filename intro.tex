With the advent of affordable devices like Oculus Go and AR-enabled smartphones, mixed reality devices are hitting the mainstream market. Market growth projections over the next five years range from 40-80\%. As we've experienced with the ubiquity of artificial intelligence (AI) and machine learning (ML), the development of ethical frameworks and guidelines tends to lag behind the technology itself. Once we begin to fully recognize the impact of new technologies, we're left to retrofit regulations and ethical decision making into technology that's already had billions of dollars invested in it.

% https://medium.com/vr-first/a-summary-of-augmented-reality-and-virtual-reality-market-size-predictions-4b51ea5e2509

Mixed reality (MR) devices blend digital elements and the physical world, covering a wide spectrum that includes both virtual reality (VR) and augmented reality (AR). In VR, a device occludes the user's vision (and often other senses) to present a fully digital experience, while AR experiences overlay digital elements on users' perceptions of the physical world. Two other terms that often appear when discussing MR are "spatial computing" or "immersive technologies."

% https://docs.microsoft.com/en-us/windows/mixed-reality/mixed-reality

The contributions of this paper are to outline the unique risks posed by MR devices, present plausible scenarios that may result from insufficient protections, and propose steps that will improve protections from legal, regulatory, societal, and engineering standpoints.



